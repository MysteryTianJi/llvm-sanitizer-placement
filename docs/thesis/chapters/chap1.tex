\chapter{绪论}

\section{研究背景}

在数字化进程不断深入的今天,软件系统已演变为国家关键基础设施、工业控制系统、企业服务平台以及个人智能设备的核心神经中枢。这一趋势使得软件安全问题日益突出,成为关乎国家安全与社会稳定的重要议题。过去二十年的安全实践表明,传统软件开发模式在内存安全防护方面存在严重不足。特别是在操作系统内核、高性能计算集群、物联网终端以及区块链底层平台等关键领域,C/C++ 等系统级编程语言仍占据主导地位。然而,这些语言设计之初缺乏原生的边界检查与自动内存管理机制,使得内存破坏漏洞(Memory Corruption Vulnerabilities)长期存在且难以根除。

内存破坏漏洞对软件安全构成了持续且严峻的威胁。据 Szekeres 等人的系统性研究统计,超过 60\% 的高危安全漏洞(High-Severity Vulnerabilities)均可归因于内存安全问题 \cite{szekeres2013eternal}。常见的漏洞类型涵盖了缓冲区溢出、堆内存破坏、格式化字符串漏洞、释放后使用(UAF)、重复释放以及未初始化内存读取等。这些漏洞通常具有以下典型特征:

\begin{enumerate}
    \item \textbf{隐蔽性强}:多数内存错误在正常执行流程中不会立即触发崩溃,往往在系统中潜伏多年,直到特定输入触发才显现;
    \item \textbf{危害性高}:攻击者可利用此类漏洞劫持控制流,进而获得系统最高权限(Root/Admin),造成数据泄露或系统瘫痪;
    \item \textbf{定位困难}:漏洞的表现现象(如崩溃)与根本原因(如内存越界写入)在时间和空间上通常距离较远,使得调试和根本原因分析(Root Cause Analysis)成本极高。
\end{enumerate}

为应对上述挑战,软件工程领域已发展出多种检测技术,包括静态分析、符号执行、模糊测试、硬件辅助检测和动态程序分析等。其中,动态检测技术因能够在真实执行路径上进行验证且漏报率较低,已成为工程实践中最重要的安全保障手段之一。特别是基于编译器的 Sanitizer 技术,能够在编译阶段自动插入运行时检测代码,无需开发者额外干预即可为内存安全提供有效的运行时保障,已成为现代软件构建链的标准配置。

\section{问题陈述与研究动机}

尽管 Sanitizer 技术在漏洞检测方面表现出色,但其在实际生产环境中的广泛应用仍面临严峻挑战,主要体现在**性能开销**、**资源占用**以及**检测可靠性**三个方面。

首先,Sanitizer 引入的插桩代码带来了显著的性能损耗。例如,工业界最常用的 AddressSanitizer (ASan) 在典型工作负载下会导致 $2\sim 3$ 倍的运行时开销和 $1.5\sim 2$ 倍的二进制体积增长 \cite{serebryany2012asan}。这种量级的开销在高性能计算或资源受限的嵌入式场景中往往是不可接受的。

其次,更为关键的问题在于**编译器优化与安全插桩之间的潜在冲突**。近期研究(如 PLDI 2023 发表的《Don't Look UB》)指出,编译器的优化 Pass 可能在 Sanitizer 插桩之后对中间表示(IR)进行激进变换,导致部分检测代码被意外移除或绕过 \cite{isemann2023dontlookub}。这种被称为 ``Sanitizer-Eliding'' 的现象直接威胁到了 Sanitizer 提供的安全保障,使得开发者误以为程序安全,实则漏洞依旧存在。

因此,\textbf{Sanitizer 的插桩位置选择}已成为决定其检测能力、运行时性能和二进制体积的关键变量。不恰当的插桩位置可能导致严重的负面后果:
\begin{itemize}
    \item \textbf{插桩过早(Too Early)}:后续的优化 Pass(如 \texttt{InstCombine}、\texttt{LoopUnroll}、\texttt{GVN}、\texttt{DCE})可能会基于未定义行为假设,改变程序语义或直接移除插桩代码;
    \item \textbf{插桩过晚(Too Late)}:此时类型信息、指针别名关系和高层中间表示结构已被破坏(Lowered),导致检测精度下降,且可能对大量死代码进行无效插桩,增加无谓开销;
    \item \textbf{LTO 阶段插桩}:虽然能利用跨模块信息进行优化,但可能受到后端优化或 \texttt{GlobalISel} 机制的限制,导致插桩失效。
\end{itemize}

当前,LLVM 框架中 Sanitizer 的插桩位置相对固定且偏向保守,学术界和工业界对于“最佳插桩位置”缺乏系统性的定量研究。尽管已有 LiteRSan \cite{xia2025litersan} 等工作尝试通过静态分析减少插桩数量,但针对编译优化流程本身的插桩位置调整研究仍处于起步阶段。

\section{研究内容与意义}

基于上述背景,本文依托 LLVM 编译器基础设施,系统研究 Sanitizer 在 Pass Pipeline 中的最佳插桩位置选择问题。本研究从编译器实现机制、系统评测和安全性验证三个维度展开,旨在构建高效、可靠的现代编译安全体系。

本文的具体研究价值体现在以下五个方面:
\begin{enumerate}
    \item \textbf{提升漏洞检测可靠性}:通过优化插桩位置,避免 \texttt{sanitizer-eliding} 现象,确保内存安全检测逻辑在经过激进优化后依然有效。
    \item \textbf{降低运行时性能开销}:探索最优的插桩时机,减少对死代码或冗余路径的插桩,预计可将运行时开销和二进制体积膨胀降低 20\%--40\%。
    \item \textbf{深化编译器优化理论}:系统探索编译器优化理论(如死代码消除、别名分析)与动态安全机制之间的相互作用关系,填补该领域的理论空白。
    \item \textbf{指导工业界实践}:为 Google、Meta、腾讯等广泛依赖 LLVM 工具链的科技企业提供具体的构建配置建议,助力其优化软件构建流程。
    \item \textbf{服务国家软件安全战略}:研究成果有助于提升自主可控软件工具链的安全性与健壮性,具有重要的现实战略意义。
\end{enumerate}