\chapter{国内外研究现状}

内存安全作为软件安全领域的核心议题,其研究历程与攻击技术、防护机制以及编译器基础设施的发展紧密相连。纵观现有研究体系,国内外关于内存安全 Sanitizer 及其与编译器优化交互作用的研究大致可分为三个阶段:内存安全漏洞的系统性分析、防护机制的技术演进,以及近年来对 Sanitizer 与编译器优化相互影响的深入探索。

本章将从上述多个维度对相关研究进展进行系统回顾,并分析现有技术的局限性。

\section{内存安全问题的长期性与复杂性}

内存安全漏洞的产生既有历史原因,也源于系统软件设计的结构性因素。Szekeres 等人在 IEEE S\&P 上发表的综述论文《SoK: Eternal War in Memory》从攻击者视角系统梳理了内存破坏漏洞的演化路径、利用技术及防护策略 \cite{szekeres2013eternal}。该研究指出,尽管现代操作系统已广泛采用地址空间布局随机化(ASLR)、栈保护(Stack Canary)、数据执行保护(DEP)等硬件与系统级防护机制,但内存破坏漏洞的数量并未显著减少,其形式反而随着防御技术的升级而变得更加隐蔽和复杂。

国内学者也对这一领域进行了深入探索。武成岗等人对 C/C++ 程序内存安全漏洞检测技术进行了全面综述 \cite{wu2018memorysafety},通过分析指出了当前面临的主要挑战:
\begin{itemize}
    \item \textbf{检测成本高昂}:高精度的分析往往伴随着指数级的时间复杂度,误报率难以有效控制;
    \item \textbf{覆盖率不足}:实际软件规模庞大,传统的静态分析技术难以覆盖所有动态执行路径;
    \item \textbf{部署困难}:动态检测技术通常带来巨大的运行时开销,阻碍了其在生产环境中的部署应用。
\end{itemize}
这些研究共同强调了高效动态检测技术(如 Sanitizer)在现代软件开发周期(SDLC)中的重要性,同时也指出其性能开销是制约实际应用的关键瓶颈。

\section{Sanitizer 技术的理论基础与工程实践}

Sanitizer 作为编译器集成的动态检测框架,自 2012 年 Google 推出 AddressSanitizer (ASan) 以来便成为安全研究的热点。ASan 创新性地采用了\textbf{影子内存(Shadow Memory)}与\textbf{红区(Redzone)}技术来检测内存越界访问。由于其高效的映射算法和便捷的部署方式(只需添加编译选项),ASan 被广泛应用于 Chrome、Android、Linux 内核等大型软件项目的测试与构建中 \cite{serebryany2012asan}。

% 这里建议之后插入一张 ASan 影子内存映射的原理图
% \begin{figure}[htbp]
%     \centering
%     \includegraphics[width=0.8\textwidth]{figures/asan_mapping.png}
%     \caption{AddressSanitizer 影子内存映射机制示意图}
%     \label{fig:asan_mapping}
% \end{figure}

随后,LLVM 社区基于类似的原理,陆续开发了多种专用 Sanitizer,构建了完整的动态检测家族:
\begin{itemize}
    \item \textbf{UndefinedBehaviorSanitizer (\texttt{UBSan})}:主要用于检测 C/C++ 标准中的未定义行为,如整数除零、无效移位、有符号整数溢出和空指针解引用等 \cite{ubsan_docs};
    \item \textbf{ThreadSanitizer (\texttt{TSan})}:专注于多线程环境下的数据竞争(Data Race)和死锁检测;
    \item \textbf{MemorySanitizer (\texttt{MSan})}:专门针对未初始化内存读取(Uninitialized Memory Read)的检测工具;
    \item \textbf{\texttt{HWASan} / \texttt{ASan-Lite}}:面向 ARM64 平台的、利用硬件辅助特性的轻量级检测方案,旨在解决移动端的性能瓶颈;
    \item \textbf{\texttt{CFISan} / \texttt{SafeStack}}:专注于控制流完整性(CFI)和栈空间安全的防护机制。
\end{itemize}

Sanitizer 的核心技术在于\textbf{编译器插桩(Compiler Instrumentation)}——即在 LLVM 中间表示(IR)层面自动插入额外的检测逻辑。因此,编译优化的顺序、插桩的具体位置以及后续优化 Pass 的行为,对 Sanitizer 的正确性和性能具有决定性影响。

\section{Sanitizer 运行时开销的实证研究}

尽管检测能力强大,但高昂的性能开销始终是阻碍 Sanitizer 大规模应用的主要障碍。Google 及学术界的多项实证评估表明:
\begin{itemize}
    \item ASan 的典型运行时开销约为 $2\times$--$3\times$,内存开销约为 $2\times$--$4\times$;
    \item MSan 的运行时开销可能达到 $3\times$--$4\times$;
    \item TSan 在并发密集型场景下,由于大量的锁竞争分析,开销甚至可达 $5\times$--$15\times$。
\end{itemize}

Vintila 等人在 2025 年 IEEE S\&P 上发表的研究从实际软件工程角度出发,系统评估了现有 Sanitizer 技术的有效性,指出高开销与高误报率共同限制了 Sanitizer 的普及 \cite{vintila2025evaluating}。该研究还发现,不当的插桩位置会进一步放大性能开销并产生大量冗余检查。例如,对死代码(Dead Code)或已被证明安全的内存访问进行插桩,是造成性能浪费的主要原因。因此,优化插桩策略成为降低开销的重要研究方向。

国内学者赵荣彩等人也系统总结了基于 LLVM 的程序分析与优化技术 \cite{wang2021llvm},指出 LLVM IR 优化流程的复杂性对任何插桩 Pass 的执行位置都有严格要求,强调了插桩策略对最终代码性能的直接影响。

\section{编译器优化对 Sanitizer 机制的干扰研究}

近年来,越来越多的研究开始关注编译器优化对安全工具的负面影响。研究发现,编译器优化不仅影响程序语义,还可能破坏 Sanitizer 的检测机制,具体表现为:
\begin{itemize}
    \item \textbf{无用代码消除 (\texttt{DCE}) 和全局值编号 (\texttt{GVN})}:可能错误地认为安全检查代码是“无用的”并将其移除;
    \item \textbf{跨基本块优化}:可能导致检查点位置发生偏差,无法保护目标指令;
    \item \textbf{循环向量化 (\texttt{LoopVectorize}) 和循环展开 (\texttt{LoopUnroll})}:改变了内存访问模式,导致原有的边界检查失效;
    \item \textbf{指令组合 (\texttt{InstCombine})}:可能修改检查代码的结构,使其语义发生变化。
\end{itemize}

Isemann 等人在 PLDI 2023 上发表的《Don't Look UB》一文中,系统阐述了 \textbf{Sanitizer-Eliding} 问题 \cite{isemann2023dontlookub}。他们通过形式化验证和实证分析,证明了某些优化 Pass 会在保持程序语义不变的前提下(基于 UB 假设),移除 Sanitizer 插桩代码,从而导致安全保障失效。这项工作首次揭示了 Sanitizer 与编译器优化之间的潜在冲突,为本课题提供了重要的理论依据。

此外,Xu 等人在 USENIX Security 2023 上的研究进一步指出,编译器自身在优化过程中也可能引入新的安全缺陷 \cite{xu2023silent},这表明安全工具不能盲目信任编译器的优化结果,需要更加谨慎地选择与优化流程的集成方式。

\section{插桩优化与 LTO 技术演进}

为解决上述问题,研究者们提出了多种优化方案。在选择性检测技术方面,LiteRSan (2025) 利用 Rust/C++ 的语义特性,通过静态分析减少不必要的插桩点 \cite{xia2025litersan};SoftBound+CETS 通过软件实现的指针元数据验证机制降低运行时负担 \cite{nagarakatte2009softbound, nagarakatte2010cets};Fat Pointer 技术则通过扩展指针结构来减少动态检查开销 \cite{zhou2023fatpointers}。然而,这些方法大多关注于插桩逻辑本身的简化,尚未系统探索“插桩位置”这一维度的优化潜力。

另一方面,链接时优化(LTO)和 ThinLTO 是 LLVM 近年来重要的演进方向 \cite{johnson2017thinlto}。LTO 能够提供全程序(Whole-Program)语义信息,为插桩优化带来了新的机遇:
\begin{itemize}
    \item 指针别名分析更加精确,可减少保守插桩;
    \item 能够实现跨模块的插桩代码去重和合并。
\end{itemize}
目前,Google 等企业已开始尝试在 LTO 阶段进行 Sanitizer 插桩以降低开销 \cite{kern2024secure}。然而,关于 LTO 插桩位置对检测能力具体影响的公开研究仍然较少,这正是本课题拟重点突破的方向。

\section{本章小结}

综合分析现有文献,可以看出当前研究在内存安全理论、Sanitizer 实现技术和编译器优化流程等方面已取得了显著进展。然而,针对 \textbf{LLVM Pass Pipeline 中 Sanitizer 插桩位置} 的系统性研究仍存在空白,特别是缺乏对不同插桩位置在性能、体积和安全性(防止优化移除)三个维度上的量化对比。本课题将致力于填补这一空白,构建一套科学的插桩位置优化框架。